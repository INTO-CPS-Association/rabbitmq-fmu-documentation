\section{Configuration}\label{sec:configuration}
This section covers how to configure an EDB FMU based on the RabbitMQ
implementation.
The data format of the messages is JavaScript Object Notation (JSON), which is
based on key/value pairs.

There are two key parts of the configuration: Configuring the data source and
configuring the FMI outputs. The data source configuration specifies connection
details related to the RabbitMQ server.
The FMI outputs configuration specifies which key/value pairs of the messages to
use.

This require changes to the FMU, but only to the static description file within
the FMU. Thus, the precompiled binary still works, since it parses the static
description file initially. Since RabbitMQ FMU has to parse the static
description file it is necessary to store a copy of the model description file
within the resources folder of the FMU.

\subsection{Configuring the Data Source}
The data source is configured via FMI parameters, as these can be set externally
before \texttt{fmi2EnterInitializationMode}\footnote{This does not hold for all
  scalar variables with parameter as causality. Check the FMI standard to ensure that you have the correct configuration.}. For this reason, it is possible to
either preconfigure the parameters in the static description file in the
resources folder of the FMU or set the values during execution of the FMU.

\Cref{lst:data-source-config} presents an example of configuring the data
source. Notice the combination of \texttt{causality="parameter"} and
\texttt{variability="fixed"} which implies that it is possible to set the values
before \texttt{fmi2EnterInitiali\-za\-tion\-Mode}, as it is necessary since a start
message is expected in the execution of \texttt{fmi2EnterInitializationMode} as
described in \cref{sec:time_handling}.
\lstset{
%  basicstyle=\ttfamily,
%  columns=fullflexible,
%  frame=single,
  postbreak=\raisebox{0ex}[0ex][0ex]
  {\ensuremath{\rcurvearrowse\space}},
  breakatwhitespace=true,
  breaklines=true,
  numbers=left,
  numberstyle=\scriptsize,
  frame=single
%  postbreak=\mbox{\textcolor{red}{$\hookrightarrow$}\space},
}
\begin{lstlisting}[label={lst:data-source-config},caption={Data source
    configuration of RabbitMQ FMU},language=XML]
<ScalarVariable name="config.hostname" valueReference="0" variability="fixed" causality="parameter">
    <String start="localhost"/>
</ScalarVariable>
<ScalarVariable name="config.port" valueReference="1" variability="fixed" causality="parameter">
    <Integer start="5672"/>
</ScalarVariable>
<ScalarVariable name="config.username" valueReference="2" variability="fixed" causality="parameter">
    <String start="guest"/>
</ScalarVariable>
<ScalarVariable name="config.password" valueReference="3" variability="fixed" causality="parameter">
    <String start="guest"/>
</ScalarVariable>
<ScalarVariable name="config.routingkey" valueReference="4" variability="fixed" causality="parameter">
    <String start="linefollower"/>
</ScalarVariable>
<ScalarVariable name="config.communicationtimeout" valueReference="6" variability="fixed" causality="parameter" description="Network read time out in seconds">
    <Integer start="60"/>
</ScalarVariable>
  \end{lstlisting}


\subsection{Configuring the FMI Outputs}
The scalar variables with \texttt{causality="output"} within the static
description are used by RabbitMQ FMU to define which key/value pairs to extract
from the messages that it receives. Thus, the name attribute of the output
scalar variable defines the key, whose paired value should be outputted as the
given scalar variable. Furthermore, the type of the value type must match the type of the
output scalar variable. An example is presented in \cref{lst:fmi-outputs-config}
where a single output of type \texttt{Real} is defined. Note, that the output
scalar variable must be added to the \texttt{Outputs} element based on index as
required by FMI, see \cref{lst:fmi-outputs-config2}. The name of the output scalar variable matches the
key/value pair \texttt{"level"} of a message such as the one exemplified in \cref{lst:message-example}

\begin{lstlisting}[label={lst:fmi-outputs-config},caption={FMI Outputs scalar
    variable configuration of RabbitMQ FMU},language=XML]
<!-- index=1 -->
<ScalarVariable name="level" valueReference="20" variability="continuous" causality="output">
    <Real />
</ScalarVariable>
\end{lstlisting}
\begin{lstlisting}[label={lst:fmi-outputs-config2},caption={FMI Outputs index
    configuration of RabbitMQ FMU },language=XML]
<Outputs>
    <Unknown index="1"/>
</Outputs>
\end{lstlisting}

\begin{lstlisting}[label={lst:message-example},caption={Example of a JSON message.},language=JSON]
{"time": 321, "level": 3.2}
  \end{lstlisting}



%%% Local Variables:
%%% mode: latex
%%% TeX-master: "../rabbitmq-fmu"
%%% End:
